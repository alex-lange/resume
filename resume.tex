\documentclass{article}

\usepackage{thumbpdf}
\usepackage[pdftex, hidelinks]{hyperref}
\usepackage{arlres2}
\usepackage{multicol}
\usepackage{fontawesome}
%\linespread{1.01}
   
\RequirePackage{soul,colortbl,color}
\usepackage{enumerate}
\usepackage{fancyhdr}
\pagestyle{fancy}

\parskip=1ex
 
\parindent=0pt 

\usepackage[top=0.7in, bottom=0.7in, left=0.8in,
  right=0.8in]{geometry}

\setmainfont[Ligatures=TeX,Scale=1.05]{Linux Libertine O}


\begin{document}

\begin{center}
  {\Large \bf Alexander R. Lange}\\[0.5em]
  \begin{multicols}{2}
    {\small 150 King St N, Waterloo, Ontario\\
      USA: (585)$\:$469-6313 $\;\bullet\;$ Canada: (226)$\:$606-2572}
    \columnbreak

    {\small
      {\tt\href{mailto:alange@uwaterloo.ca}{alange@uwaterloo.ca}}
      $\;\bullet\;$
      {\tt\href{http://www.alexlange.co/}{alexlange.co}}}\\
    \faLinkedinSign 
    {\small
      $\;${\tt\href{https://www.linkedin.com/in/alexrlange}{alexrlange}} 
      $\;\bullet\;$}
    \faGithub 
    {\small $\;${\tt\href{https://github.com/arlange}{arlange}} $\;\bullet\:$}
    \faTwitter 
    {\small
      $\;${\tt\href{https://twitter.com/alalexlalange}{@alalexlalange}}}
  \end{multicols}
\end{center}

\vspace*{-1.5em}

\ressec{Education}

\vspace*{-0.25em}

\resentry{\bf University of Waterloo}{Waterloo, ON}
\resentry{MMath, Combinatorics and Optimization}{May 2015}
\begin{reslist}
  \item Thesis: \emph{Approximation Algorithms for Graph Protection Problems}
  \item Concentration: Discrete Optimization, Approximation Algorithms
\end{reslist}

\vspace*{0.25em}

\resentry{\bf \rit}{Rochester, NY}
\resentry{M.S. Computer Science}{August 2013}
\begin{reslist}
  \item Thesis: \emph{Solving Hard Graph Problems with Combinatorial
    Computing and Optimization}
 \item GPA: 4.0, concentration: Theoretical Computer Science
\end{reslist}

\resentry{B.S. Computational Mathematics}{May 2011}

\vspace*{-0.5em}

%\emph{Areas of Interest:} algorithms, combinatorial computing,
%discrete optimization, graph theory

\ressec{Experience}

\vspace*{-0.25em}

\resentry{{\bf Google}}{Mountain View, CA}
\resentry{{\it Software Engineer in Test Intern}}{September 2014 --
  December 2014}
As part of the \textsf{Ads Backend Engineering Productivity} team, I worked on the following
projects (all coding in \textsf{Python}):
\begin{reslist}
  \item Release automation for Mesa, the large, distributed data
    warehousing system for all advertising statistics. This system
    has over 20 binaries and 5 release groups. Main challenges
    included learning internal tools and libraries, and designing a
    workflow that met both the requirements of the release groups and
    the needs of the engineers.
  \item Improving engineer's workflow for a submission process with
    high contention affecting multiple teams. The problem escalated
    during my internship. Attended design meetings and developed tools
    for the first stages of the solution, including the automation of
    generating new golden files and diff reporting them against 
    old ones.
\end{reslist} 

\vspace*{0.25em}

\resentry{{\bf University of Waterloo}}{Waterloo, ON}
\resentry{{\it Research Assistant}}{September 2013 -- August 2014}
Assisting Prof. Chaitanya Swamy on the design of approximation algorithms
for stochastic optimization problems, including a problem involving
partitioning a graph to protect valuable nodes against stochastically located
threats. 

%\resentry{{\it Teaching Assistant}}{September 2013 -- present}
%Ran tutorials, managed undergraduate markers, and marked assignments
%for MATH 135: Algebra for Honours Math, CO372: Portfolio Optimzation, and ECE 103: Discrete
%Math for Engineers.

\vspace*{0.25em}
 
\resentry{{\bf \rit}}{Rochester, NY} \resentry{{\it Research
    Assistant}}{January 2012 -- August 2013} Assisted Prof. Stanis{\l}aw
Radziszowski in research involving computational Ramsey
Theory. Developed a robust C++ library to generate, enumerate,
manipulate, and test graphs for a variety of Ramsey properties. Led to
publications [1,2], the work of the latter made use of the Open
Science Grid for approximately 200,000 CPU hours of computation.

\vspace*{0.25em}

\resentry{{\bf Bell and Howell / BCC Software}}{Rochester, NY} \resentry{{\it
    Software Engineer Intern}}{June 2013 -- September 2013} Researched
XML specification for company's new product feature, and developed
a prototype of the feature in Python, making use of schema bindings
and web services.  Documented all data and information as part of
design document.

\resentry{\textit{Software Test Specialist/Quality Assurance
    Intern}}{June 2010 -- November 2011}
\begin{reslist}
  \item Fully responsible for data and data quality
    testing. Participated in the development of a new procedure that
    improved quality metrics by producing and processing reports
    automated through scripts and spreadsheets.
%  \item Lead tester of new feature: created and followed a test plan,
%    tested change requests, participated in design reviews.
  \item Developed a new automated regression test library in Python
    and shell scripts for specific customer use cases.
%  \item Various testing responsibilities, including unit, performance, and
%    installation testing.
\end{reslist}

\ressec{Computer Skills}

\begin{reslist}
\item Proficient in Java, C++, Python, \LaTeX, {\sc Matlab}
\item Experience with C\#, SQL, .NET, Visual Studio, Mono, MIPS Assembly, shell scripting
\end{reslist}

\vspace*{-0.5em}

\ressec{Academic Awards and Achievements}
\begin{reslist}
  \item RIT Computer Science Outstanding Graduate Student Award, 2012--2013
  \item Placed 3rd out of 12 in the {\bf ACM-ICPC} Northeast North America
Regional Finals, Fall 2012
  \item Golisano College of Computer and Information Sciences Research
    Seed Funding, 2012 and 2013
  %\item RIT Computer Science Graduate Delegate, nominated, Fall 2012
  \item RIT Presidential Scholarship, 2007--2011; RIT Honors Program,
    2007--2010%; RIT Dean’s List, 2007--2011
\item {Publications}:\\[-1.75em]
\begin{enumerate}{\footnotesize \renewcommand{\theenumi}{[\arabic{enumi}]}
  \renewcommand{\labelenumi}{\theenumi}
  {\partopsep=0pt \topsep=0pt \leftmargin=0em \parskip=0pt}
  \setlength{\labelsep}{2pt}
  \setlength{\itemsep}{-3pt}
  \item A. Lange, S. Radziszowski, and X. Xu. Use of MAX-CUT for Ramsey
    Arrowing of Triangles. \emph{Journal of Combinatorial Mathematics
      and Combinatorial Computing}. \textbf{88} (2014) 61--71.
  \item A. Lange, I. Livinsky, and S. Radziszowski. Computation of the
      Ramsey Numbers $R(C_4,K_9)$ and $R(C_4,K_{10})$. \emph{accepted}.}
\end{enumerate}
\end{reslist}





%% \item {\bf Conference Presentations}: {\footnotesize 25th Cumberland Conf. on
%%   Combinatorics, Graph Theory, \& Computing, Johnson City, TN, May
%%   2012.  MIdwestern GrapH TheorY (MIGHTY) LIII Conference, Ames, IA,
%%   September 2012.  44th SE International Conf. on Combinatorics, Graph
%%   Theory, \& Computing, Boca Raton, FL, March 2013.}


\end{document}
